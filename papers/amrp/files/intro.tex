\section{Introduction}
\label{sec:intro}

Image segmentation refers to the partition of an image into a set of regions representing  meaningful areas. It is considered a challenging semantic task aiming to determine and group uniform regions for analysis. {\color{green}reescrito Felipe}According to~\cite{DOMINGUEZ}, to create an adequate segmented image it is necessary that the output presents some fundamental characteristics, such as: (i) region uniformity and homogeneity in  its features, aka. gray level, color, or texture; (ii) region continuity, without any holes; (iii) significant difference from adjacency regions; and (iv) spacially accurate with smothness, without raggedness. Segmentation is an active topic of research and in a traditional approach the task is performed using hand-engineered features \cite{Segnet:2017:7803544}.  

\textbf{Insert scale-set theory reference to characterize hand-eng. Talvez retirar o paragrafo do deep para introduzir o related}

\textbf{
Nas décadas 1990 e 2000, técnicas mais complexas surgiram, usando \textit{features} criadas manualmente \cite{RCF:8100105}. Trabalhos como \citeonline{arbelaez2011}, \citeonline{KONISHI:1159946} e \citeonline{MARTIN:1273918}, seguiram essa tendência e desenvolveram métodos que utilizavam informações como gradiente, intensidade de cores e texturas para detecção de bordas \cite{RCF:8100105}.
}
\textbf{
Na década atual, ainda foram propostos trabalhos com \textit{features} desenvolvidas manualmente, como \citeonline{LIM:6619250}. No entanto, novas abordagens surgiram utilizando métodos de aprendizado de máquinas para detecção de bordas. Em um novo trabalho, \citeonline{StructuredEdges:2015} propuseram o método \textit{Structured Edges}, utilizando florestas de decisão aleatórias. As redes HED, proposta por \citeonline{Xie:2017:HED:3158436.3158453}, RCF, proposta por \citeonline{RCF:8100105}, e COB, proposta por \citeonline{COB:7917294}, utilizam redes neurais convolucionais profundas para detecção de bordas.
}

Recently, deep learning architectures drastically changed the computational paradigm for visual tasks. The main advantage of deep learning algorithms is that it does not require an engineered model to operate, meaning that they are capable of learning not only the features to represent the data but also the models to describe it~\cite{goodfellow16}. The success of these approaches relies on a hierarchy of concepts learned through the network, in which more complex concepts are build from simpler ones. In the deep learning approach applied in images, the raw pixel on the input layer is learned as segments and parts until the composition of multiple object concepts later in the network.

Unsurprisingly, many approaches have been proposed to explore these hierarchies, creating maps from the outputs of different layers of a deep learning network. One challenge in this strategy is how to combine these maps, considering that they are presented with different sizes and could represent different concepts. In this work it is presented strategies to combine hierarchical maps to create region proposition for the task of binary image segmentation.   

The remainder of this work is organized as bla bla bla....

