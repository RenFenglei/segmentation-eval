\section{Conclusion}
\label{sec:conclusion}

This work addressed the problem of merging side-outputs extracted from the convolutional layer model VGG to create region propositions for the task of image segmentation. We compare 3 different merging strategies to combine the side results: $add()$, $avg()$ and $max()$. The functions to enhance were evaluated using the cross-entropy accuracy and the pixel-error rate. The impact of the number of side-outputs was studied and compared to a version without any side-outputs for a similar network architecture. At last, a simple mathematical morphology operation was proposed to  enhance the performance on the task and remove some noise. 

Experiments demonstrated that the $avg()$ function is viable for merging side-output maps with different sizes and content, and could place the proposed strategy among the state-of-the-art approaches for the task on the Kitti dataset. The use of $avg()$ merging strategy was adopted before in \cite{liu2017}, but no explanation was given to the use of this method over other possible merging strategies. This paper helps to explain the good results achieved in this paper.

It was also demonstrated that a large amount of side-outputs increases the network capabilities during the training step and could also creates jumps that could lead to better performance, in terms of accuracy. The training graphs also show that the number of side-outputs contributes to a faster decay in loss function and more stable results. The post-processing strategy slightly improves the performance, but requires further studies.

This research opens novel opportunities for study such as: (i) exploring different merging functions, less sensitive to fluctuations in values;  (ii) exploring regularization techniques to sustain larger amounts of side-outputs consistent; and (iii) insert the mathematical morphology kernels in the learning process to search for the best kernel size. 
 
The code and the list of dependencies to reproduce the experiments (under Anaconda environment) are publicly available online in \url{https://github.com/falreis/segmentation-eval}. 

\section{Acknowledgements}
\label{sec:acknowledgements}

This paper acknowledges Github repositories \url{https://github.com/lc82111/Keras_HED} and \url{https://github.com/moabitcoin/holy-edge} that were helpful to provide some basic source codes used in this work.
